\documentclass[twocolumn]{article}

\usepackage{import}
\usepackage{xcolor}
\usepackage{cancel}
\usepackage{mathtools}

% Colors

\definecolor{yorhabg}{HTML}{131314}
\definecolor{yorhafg}{HTML}{C9C7CD}
\definecolor{yorhagrid}{HTML}{B5AF9C}
\definecolor{mred}{HTML}{D67069}
\definecolor{mblue}{HTML}{6887A1}

\pagecolor{yorhabg}
\color{yorhafg}

%Importing the Preamble
\import{./}{preamble.sty}

% Removes padding above title
\usepackage{titling}
\setlength{\droptitle}{-10em}

% Font package
\usepackage[T1]{fontenc}

\usepackage{fouriernc}

\usepackage{sectsty}
\usepackage{graphicx}
\usepackage{amsmath}
\usepackage{amsfonts}
\usepackage{amssymb}
\usepackage[skins, most]{tcolorbox}

\DeclareMathOperator{\sgn}{sgn}

\usepackage{tikz}
\usepackage{eso-pic}
\usetikzlibrary{calc,shadows.blur}
\usetikzlibrary{angles, quotes}
\usetikzlibrary{3d}

% Margins
\topmargin=0in
\evensidemargin=0in
\oddsidemargin=0in
\textwidth=6.5in
\textheight=9.0in
\headsep=0.25in

\AtBeginEnvironment{tcolorbox}{\small}

\newtcolorbox{imp}{enhanced,arc=0mm,colback=yorhabg,colframe=mred,leftrule=10mm,coltext=yorhafg,%
overlay={\node[anchor=west,outer sep=2pt] at (frame.west) {\includegraphics[width=6mm]{images/imageb.png}}; }}

\newtcolorbox{shortcut}{enhanced,arc=0mm,colback=yorhabg,colframe=mred,leftrule=10mm,coltext=yorhafg, coltitle=yorhabg, title=\texttt{Shortcut.},
overlay={\node[anchor=west,outer sep=2pt] at (frame.west) {\includegraphics[width=6mm]{images/imageb.png}}; }}

\newtcolorbox{question}{
  enhanced,
  colback=yorhabg,
  colframe=mblue,
  coltext=yorhafg,
  coltitle=yorhabg,
  attach boxed title to top left={yshift*=-\tcboxedtitleheight},
  title=\texttt{Definition},
  boxed title size=title,
  boxed title style={%
    rounded corners=northeast,
    rounded corners=northwest,
    colback=tcbcolframe,
    boxrule=0pt,
  },
  underlay boxed title={%
    \path[fill=tcbcolframe] (title.south west)--(title.south east)
    to[out=0, in=180] ([xshift=5mm]title.east)--
    (title.center-|frame.east)
    [rounded corners=5pt] |-
    (frame.north) -| cycle;
  },
}

\newcommand\bb[1]{\textcolor{yorhafg}{\textbf{#1}}}

\title{\textbf{Symmetric Encryption }}
\author{ Raphael Pertler }
\date{\today}

\begin{document}

\maketitle

\section*{Encryption Schemes}

\begin{question}
  An Encryption Scheme is a tuple \( (X,K,E,D) \) with:
  \begin{itemize}
    \item \(X \subseteq \{0,1\}^*\) \ Plaintexts
    \item \(K \subseteq \{0,1\}^*\) \ the finite Set of keys
    \item \(E\) is a probabilistic encryption algorithm with \(x \in X ; k \in K\) as inputs, so that \(E (x,k) = y \in \{0,1\}^*\)
    \item \(D\) is a deterministic decryption algo. with \(y \in \{0,1\}^* ; k \in K\) as inputs and returns \(x\in X\)
  \end{itemize}
  The Scheme has to satisfy the \\ "perfect correctness" property: \\
  \(\forall x \in X ; k \in K : D(E(x,k)k) = x \) \\ \\

  \(y := E(x,k)\) is called a cyphertext.

  \(Y\) is the set of all possible cyphertexts. \\
  \(Y \subseteq \{0,1\}^*\)
\end{question}

\begin{question}
  Let \(X,K,E,D\) be an encryption scheme with deterministic encryption. \\ For \(
  k \ in K \) the function: \\ \[ E(\cdot,k) : X \rightarrow Y ; x \rightarrow E(x,
  k) \] Is called a Cipher.
\end{question}
\end{document}